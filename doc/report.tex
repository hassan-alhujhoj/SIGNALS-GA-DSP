\documentclass[a4paper, 11pt]{article}
\usepackage{graphicx}
\usepackage{url}
\usepackage{hyperref}
\usepackage[numbers]{natbib}
\usepackage{tabularx}
\usepackage{amsmath}
\usepackage{amssymb}
\usepackage{epstopdf}
\usepackage{inputenc}
\usepackage{geometry}
\usepackage{alphabeta}
\usepackage{float}
\usepackage{changepage}
\usepackage{lipsum}

\setlength{\oddsidemargin}{0mm}
\setlength{\evensidemargin}{-14mm}
\setlength{\marginparwidth}{0cm}
\setlength{\marginparsep}{0cm}
\setlength{\topmargin}{2mm}
\setlength{\headheight}{0mm}
\setlength{\headsep}{0cm}
\setlength{\textheight}{240mm}
\setlength{\textwidth}{168mm}
\setlength{\topskip}{0mm}
\setlength{\footskip}{10mm} 

\newcommand{\code}[1]{\texttt{#1}}
\newcommand{\refsec}[1]{\mbox{Section~\ref{sec:#1}}}
\newcommand{\refapp}[1]{\mbox{Appendix~\ref{sec:#1}}}
\newcommand{\refeqn}[1]{\mbox{(\ref{eqn:#1})}}
\newcommand{\reffig}[1]{\mbox{Figure~\ref{fig:#1}}}
\newcommand{\ud}{\mathrm{d}}                    % upright d (derivative)

\newcounter{foo}
\newcounter{bar}

\title{
    ENEL420 - Genetic Algorithms in Digital Signal Processing\\
    \vspace{1cm}
    \begin{large} 
        Department of Electrical and Computer Engineering\\
        University of Canterbury\\
    \end{large}
    \vspace{1cm}
}

\author{
    \small {Luke Trenberth (ID: 47277086)}\\
    \small {Hassan Alhujhoj (ID: 35352633)}\\
    }
\vspace{2cm}

\date{\small\today}
\begin{document}
\maketitle

\begin{abstract}
    I would like to say some bullshit here that summaries my report in an interesting way.
\end{abstract}

\pagebreak
\pagenumbering{roman}
\tableofcontents
\pagenumbering{arabic}
\pagebreak

\section{Introduction}\label{sec:intro}
    Genetic Algorithms (GA) are inspired by the mechanism of natural selection where the strongest and fittest
    individuals would likely be the winners in a competing environment. Genetic Algorithm is used as a direct 
    analogy of such natural evolution where it presumes that a potential solution of a problem is an individual 
    and can be represented by a set of parameters. These sets of parameters are regarded as the genes of a chromosome 
    and can be structured by a string of values in binary form. A fitness value is used to reflect the degree of 
    goodness of the chromosome for the problem which would be highly related with its objective value \cite{Man1997}.
    \\\\
    History has shown that the fitter chromosome tends to yield good quality offspring which means a
    better solution to the problem. Practically, a population pool of chromosomes must be randomly set initially. 
    The size of this population varies from one problem to the other. Each cycle of genetic operation is termed as 
    an evolving process where a subsequent generation is created from the chromosomes in the current population.  
    This evolving process can only be succeeded if a group of those chromosomes, which are generally called
    “parents” or a collection term “mating pool” are selected. The genes of the parents are then mixed to produce 
    offspring in the next generation. From this manipulation of genes process, the “better” chromosome will create a 
    larger number of offspring, and thus has a higher chance of survival in the subsequent generation, emulating the 
    survival-of-the-fittest mechanism in nature \cite{Man1997}.
    \\\\
    To make sure the desired termination criterion is reached, the cycle of evolution is repeated. The offspring of 
    the previous generation are reinserted into the model, further yielding higher quality offsprings \cite{Man1997}.
    \\\\
    There are two fundamental operators that facilitate the evolution cycle: Crossover and Mutation. Both operators 
    are required for such a process even though the selection routine. To further illustrate the crossover procedure, 
    the one-point crossover mechanism is shown in Figure \ref{Fig:crossover}. Genes are exchanged between parents to form 
    offspring. Mutations are randomly generated after crossover with a small probability of occurrence.
    
    \begin{figure}[h!]
        \centering
        \graphicspath{{./wiki/}}
        \includegraphics[scale=0.7]{crossover.png}
        \caption{Interference frequencies present in the ECG signal.}
        \label{Fig:crossover}
    \end{figure}

\section{Background}\label{sec:bg}
    \subsection{Digital Signal Processing of ECG Signals}\label{sec:bg_sub1}
        In assignment one, a noisy ECG signal with 1024Hz sampling frequency was provided to be filtered. 
        The assignment required the implementation of a notch filter with either an FIR or IIR filter.
        An FIR or IIR notch filter was suited to filter this ECG signal since there were a clear two 
        interference frequencies present within the frequency spectrum of the ECG signal.
        These interference frequencies were identified to be $f_{1} = 31.456Hz$ and $f_{2} = 74.36Hz$
        as shown in Figure \ref{Fig:rejFreq}. Interference frequencies from other ECG signals can be found to
        be between $30Hz \leqslant f \leqslant 100Hz$. It should be noted that the first peak in Figure \ref{Fig:rejFreq} is the 
        DC component due to the use FFT to get the frequency response of the time domain ECG signal.
        
        \begin{figure}[h!]
            \centering
            \graphicspath{{./wiki/}}
            \includegraphics[scale=0.6]{ECG_freq_spectrum.png}
            \caption{Interference frequencies present in the ECG signal.}
            \label{Fig:rejFreq}
        \end{figure}

        One method to filter these two frequencies was to reject them with either a window or Parks-McClellan filters.

    \subsection{Genetic Algorithms}\label{sec:bg_sub2}
        \begin{figure}[h!]
            \centering
            \graphicspath{{./wiki/}}
            \includegraphics[scale=0.8]{GA_cycle.png}
            \caption{Genetic Algorithms cycle. Adapted from \cite{Tang1996}}
            \label{Fig:GA_cycle}
        \end{figure}

        \subsubsection{Crossover}
            Crossover is a GA operator which is a recombination operator that combines subparts of two parents 
            chromosomes to produce offsprings that contain some parts of both parents' genetic material. Crossover
            is considered by many GA practitioners to be the determining factor that distinguishes the GA from
            all other optimisation algorithms \cite{Tang1996}.
        \subsubsection{Mutation}
            Mutation is another operator that introduces variations into the chromosomes. This variation can be
            local or global. Mutation can occur some occasionally but can randomly alters the value of a string
            position. A randomly generated bit can replace any bit of the chromosome bitstring mutating the original
            bit sequence of the parents \cite{Tang1996}.
        \subsubsection{Parents}
        
        \subsection{Applications}
        Since the emergence of Darwin's theory of natural selection, GA has become a powerful tool and used in various
         applications.  The GAs are also known as optimisation algorithms. Optimisation algorithms have two major classes.
         These two classes are classified as calculus-based techniques and enumerative techniques. Calculus-based
         optimisation algorithms employ the gradient-directed searching mechanism to solve error surfaces or differentiable
         surfaces of an objective function. A common misuse of an objective function can occur with an ill-defined or multimodal
         objective function. This can lead to obtaining a local optima instead of a global optima. Such use of objective
         functions are common in signal processing \cite{Tang1996}.
         \\\\
         Genetic Algorithms have a general cycle known as GA cycle, shown in Figure \ref{Fig:GA_cycle}. This cycle is a searching
         process  based on the laws of natural selection and genetics. A simple GA consists of three operations: Selection, Genetic
         Operation and Replacement. GA population comprises of a group of chromosomes from which candidates can be selected for the
         solution of a problem. The population, initially, is generated randomly. The fitness values of all chromosomes are 
         evaluated by calculating the objective function in a decoded form (phenotype). to generate the offspring by the defined 
         genetic operations, a particular group of chromosomes (parents) are selected from the total population. The fitness of the 
         offspring is evaluated in a similar fashion to their parent chromosomes. The chromosomes in the current population are then 
         replaced by their offspring based on a certain replacement strategy defined by the user \cite{Tang1996}.
         \\\\
         This GA cycle is then repeated until a desired termination criterion is reached. For example a predefined number of generations
         is produced. If all goes well and according to this process of simulated evolution, the best chromosomes is the final population 
         can become a highly evolved solution to the problem. Generally, GA follows the following process \cite{Tang1996}.

         \begin{enumerate}
            \item Randomly generate an initial population $X(O):= (x_{1}, x_{2}, ..., x_{N})$.
            \item Compute the fitness $F(x_{i})$ of each chromosome $x_{i}$ in the  current population $X(t)$;
            \item Create new chromosomes $X_{r}(t)$ by mating current chromosomes, applying mutation and recombination (crossover) as the 
            parent chromosomes mate;
            \item Delete numbers of the population  to make room for the new chromosomes;
            \item Compute the fittness of $X_{r}(t)$, and insert these into the popuation;
            \item $t:= t + 1$, if not (end-test) go to step 3, or else stop an dreturn the best chromosome.
        \end{enumerate}

        Another application is...

\section{Method}\label{sec:meth}
    To apply genetic algorithms to Digital Signal Processing, it was decided to use genetic algorithms to design a 
    Finite Impulse Response filter. Input parameters of the transition width of bands, the bandwidth
    and the filter frequencies were used to create a filter. A fitness function was then chosen to analyse the fitness
    of each sample in the population. By selecting the parents with the highest fitness and conducting genetic crossover,
    variation and mutation. Through iteration of this process, the population fitness increased, giving the optimal results, 
    identifying the optimal parameters to identify the most optimal filter, and the best result was returned for the user.

    \begin{figure}[h!]
        \centering
        \graphicspath{{./wiki/}}
        \includegraphics[scale=1]{Flowchart.png}
        \caption{Flowchart diagram of the use of GA to process ECG signals.}
        \label{Fig:flowchart}
    \end{figure}

    \subsection{Digital Signal Processing Data}\label{sec:meth_sub1}
        Due to easily accessible data samples, the ECG signals from Assignment 1 have been analysed. This data contained
        two interference frequencies between  $f = 30 - 100Hz$. The interference frequencies must be identified and 
        removed from the data.
        \\\\
        A python class was created to analyse the signal. This class was able to convert the time domain signal to the
        frequency domain, giving a plot of the signal. This class was able to conduct filtering techniques such as Window
        Filtering, Frequency Sampling and Parks-McLellan Filtering. Plots of filtered and unfiltered data was generated. 
        This allowed the data to be analysed with a level of abstraction. 

    \subsection{Fitness Function}\label{sec:meth_sub2}
        For Genetic Algorithms, it was important to identify which samples were more optimal than others. A fitness
        function must be identified to identify more useful results. For digital signal processing, the Signal to Noise
        Ratio (SNR) is considered a useful metric for choosing the most optimal filtering coefficients. For each 
        individual, a Parks-McLellan filter was generated, and then applied to the original data. The filtered signal
        power was then calculated by considering the variance in the signal. The difference between the filtered power
        and the original signal power was then considered the noise power, and the signal to noise power was calculated. 
        Signals with higher SNR's were considered more optimal filtering. Due to this, the signal to noise ratio of the
        signal was determined to be an appropriate measure of the sample's fitness. 

        \begin{equation} SNR = var(y_0) - var(y)\end{equation}

    \subsection{Population Selection}\label{sec:meth_sub3}
        Parent chromosomes were chosen by taking the chromosomes with the largest fitness scores in the previous generation. 
        Chromosomes with the largest fitness scores have the most successful genes and were incorporated into the
        successor generation gene pool. For the analysed ECG data set, a population of 20 chromosomes was used, with four
        parents succeeding to the following generation. The effect of modifying each of these results on the success of the
        algorithm and the number of iterations for success was investigated. 
        \\\\
        Successive chromosomes were created through a combination of genetic mutation, and crossover of the parent chromosomes.
        Genetic mutation was incorporated into the genetic algorithm by random variation in the genetic sequences, using numpy's 
        randint function. This allowed the frequency genes to span the entire frequency range. Genetic crossover was implemented
        to find optimal combinations of known genes in the gene pool. This mixes genes from parent chromosomes. The effectiveness of 
        found gene combinations was assessed in the next generation, ideally finding combinations of chromosomes that yield higher 
        fitness. Any combinations with higher fitness than the original are then given a higher selection priority. 

    \subsection{Filtering of ECG Signal Using FIR Filters}\label{sec:meth_sub4}
        \subsubsection{Window Filter}
            Plz write something here.

        \subsubsection{Parks-McClellan Filter}
            Plz write something here.


\section{Results}\label{sec:res}

    \begin{figure}[h!]
        \centering
        \graphicspath{{./wiki/}}
        \includegraphics[scale=0.8]{af_filtering_1kGen20Pop.png}
        \caption{The filtered output of ECG signal 1, using $p_size = 20$, $n = 1000$.}
        \label{Fig:result_1}
    \end{figure}
    
    When the Genetic Algorithm was run, the algorithm was able to identify the optimal filter frequencies for maximum SNR.
    With the filter  population size of $p_size = 20$, and a maximum number of iterations of $n = 100$, a filtered output
    was produced. As seen in Figure \ref{Fig:result_1}.
    
    \subsection{Creating a Population from DSP Singal Data}\label{sec:meth_sub1}
    Plz write something here.
    \subsection{Fitness Function}\label{sec:meth_sub2}
    Plz write something here.
    \subsection{Selecting Paranets for next GA Generations}\label{sec:meth_sub3}
    Plz write something here.
    \subsection{GA Operators}\label{sec:meth_sub4}
    Plz write something here.
        \subsubsection{Crossover}
            Plz write something here.
        \subsubsection{Mutation}
            Plz write something here.
        \subsubsection{Parents}
            Plz write something here.
    \subsection{Filtering of ECG Signal Using FIR Filters}\label{sec:meth_sub5}
        \subsubsection{Window Filter}
            Plz write something here.
        \subsubsection{Parks-McClellan Filter}
            Plz write something here.

\section{Discussion}\label{sec:dis}
    Discussion.
    \subsection{Improvements}
        Plz write something in here.

\section{Conclusion}\label{sec:conc}
    Reinstate the stuff you've talked about in the report. Don't introduce new materials in here.

\pagebreak
\bibliographystyle{IEEEtran}
\renewcommand{\bibname}{References}
\renewcommand{\bibsection}{\section{\bibname}}
\renewcommand{\cite}{\citep}
\bibliography{ref}
\pagebreak

\end{document}

